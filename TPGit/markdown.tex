\hypertarget{objectifs}{%
\section{Objectifs}\label{objectifs}}

\begin{itemize}
\tightlist
\item
  Apprendre à se servir d'un Logiciel de Gestion de Version.
\item
  Matrisez le versionnement d'un projet logiciel.
\item
  Partager un projet et travailler en équipe.
\end{itemize}

\hypertarget{help}{%
\section{Help}\label{help}}

This text is \emph{emphasized with underscores}, and this is
\emph{emphasized with asterisks}.

This is \textbf{strong emphasis} and \textbf{with underscores}.

This is * not emphasized *, and *neither is this*.

feas\emph{ible}, not feas\emph{able}.

This \sout{is deleted text.}

H\textsubscript{2}O is a liquid. 2\textsuperscript{10} is 1024.

\emph{Italic} \ULforem \emph{Underline} \normalem

What is the difference between \mintinline[]{text}{>>=} and
\mintinline[]{text}{>>}?

Here is a literal backtick \mintinline[]{text}{`}.\\

This is a backslash followed by an asterisk: \mintinline[]{text}{\*}.\\

\textsc{Small caps}

\textsc{Small caps}

\textsc{Small caps}

\begin{minted}[]{java}
	if (a > 3) {
		moveShip(5 * gravity, DOWN);
	}
\end{minted}

\begin{minted}[]{haskell}
qsort []     = []
qsort (x:xs) = qsort (filter (< x) xs) ++ [x] ++
qsort (filter (>= x) xs)
\end{minted}

\leavevmode\hypertarget{special}{}%
Here is a paragraph.

And another.

\url{http://google.com}

\begin{figure}
\centering
\includegraphics{images/information.pdf}
\caption{This is the caption}
\end{figure}

\hypertarget{level-two}{%
\subsection{Level two}\label{level-two}}

\$-- An inline
\includegraphics[width=0.3125in,height=0.3125in]{images/information.pdf}
and a reference \includegraphics{images/information.pdf} with
attributes.\\

\begin{itemize}
\tightlist
\item
  line
\item
  line
\end{itemize}

and a reference \includegraphics{images/information.pdf} with
attributes.\\

\begin{minted}[]{java}
<div class="container">
    <div class="row">
        <div class="col-md-4">
			<img src="/img/wip3.jpg" alt="Work in progress"/>
		</div>
        <div class="col-md-7">
			<img src="/img/nslu2.jpg" alt="NSLU2.jpg"/>
        </div>
    </div>
</div>
\end{minted}

\begin{center}\rule{0.5\linewidth}{\linethickness}\end{center}

\begin{minted}[]{html}
<div class="row">
	<div class="col-md-4">
		<img src="/img/wip3.jpg" alt="Work in progress"/>
	</div>
	<div class="col-md-7">
		<img src="/img/nslu2.jpg" alt="NSLU2.jpg"/>
	</div>
</div>
\end{minted}

\hypertarget{howdie}{%
\subsection{Howdie}\label{howdie}}

It's easy to setup additional IP addresses on Debian Linux. This is
particularly useful for the NSLU which doesn't have a display so you
need to remotely connect to it.

Having more than one virtual network interface allows us to have both a
DHCP address and a static IP address, making the NSLU2 accessible
\texttt{pretty much always}.

\hypertarget{on-the-nslu}{%
\subsubsection{On the nslu}\label{on-the-nslu}}

I'm supposing you have somekind of IP address already setup. Also, you
should have a working DNS, although this shouldn't impact any of the
following.

Once you can ping the gateway machine, the gateway must be added so that
the nslu knows that he can attain addresses that are beyond his own
network.

Command is

\begin{minted}[]{bash}
	route add default gw 192.168.0.100
\end{minted}

Then the \mintinline[]{text}{/etc/resolv.conf} must be edited to the
same nameserver as the other box. Note that nslu should be able to ping
this nameserver once the gateway is setup.

For setting up the gateway we must enable ip forwarding:

\begin{minted}[]{bash}
echo 1 > /proc/sys/net/ipv4/ip_forward
\end{minted}

\hypertarget{additional-virtual-network-interface}{%
\subsubsection{Additional virtual network
interface}\label{additional-virtual-network-interface}}

If we want to setup alias cards, we can do something along the lines of

\begin{minted}[]{bash}
ifconfig eth0:N address    netmask
\end{minted}

However, it's probably easier to just edit the
\mintinline[]{text}{/etc/network/interfaces} file by adding something
like this

\{\% highlight bash\%\} iface eth0:0 inet static address 192.168.0.100
netmask 255.255.255.0 broadcast 192.168.0.255 network 192.168.0.0 pre-up
echo ``*** Starting eth0:0 alias ***" post-up echo ``*** Alias eth0:0
started ***" \#post-up route add -net 192.168.0.0 dev eth0 \#post-up
route add -host 192.168.0.10 dev eth0:1 \{\% endhighlight \%\}

This allows us to have a static IP address in addition to whatever other
configured addresses we might have. I used it to have both a DHCP
assigned address and a static local network address, allowing me to
always get access to the NSLU2, even if it's not getting an address
through DHCP.

Finally, must set up iptables to the correct rules!!!

I found and change a script to setup the tables to allow
