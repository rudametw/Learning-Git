% UNCOMMENT FOLLOWING LINE FOR TESTING PURPOSES
%
%\newcommand{\CommandLineArg}{../CM1/CM1-Intro.pdf}

\def\ReadCommandLineArg#1 {%
  \def\CommandLineArg{#1}%
  \input{\jobname}}
\unless\ifdefined\CommandLineArg
\endinput\expandafter\expandafter\expandafter\ReadCommandLineArg\fi


% If it wasn't defined, let's define the file to compile
%\ifx\CommandLineArg\undefined
%	\newcommand{\CommandLineArg}{../CM1/CM1-Intro.pdf}
%\fi
%\else
%  %% Do this if it is defined

%\fi



\documentclass[a4paper]{article}
\usepackage{pdfpages}

\setlength{\fboxrule}{0.01pt}

\begin{document}
% https://wiki.bath.ac.uk/display/latextricks/Making+handouts+from+your+beamer+presentation
%4 slides per page
%\includepdf[pages=1-last,nup=1x2,landscape=true,frame=true,noautoscale=true,scale=1,delta=5mm 5mm]{\CommandLineArg}
\includepdf[pages=1-last,nup=1x2,scale=1.05, delta=0mm -18mm,landscape=true]{\CommandLineArg}

%6 slides per page
%\includepdf[pages=1-last,nup=2x3,landscape=false,frame=true,noautoscale=true,scale=0.75,delta=2mm 20mm]{CM2-structures.pdf}
%\includepdf[pages=1-last,nup=2x3,landscape=false,frame=true,noautoscale=true,scale=0.75,delta=2mm 22mm]{\CommandLineArg}

\end{document}
